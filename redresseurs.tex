\section{Redresseurs}
\renewcommand{\arraystretch}{1.25}
\begin{tabular}{ll}
    Tension pointe à l'entrée & \( V_{2p}= \sqrt{2}*V_{RMS} * N_2/N_1\)\\
    Si la diode est idéale & \(V_{\mathrm{seuil}}=0\)
\end{tabular}

\subsection{sans filtrage}
\subsubsection{Simple alternance}
\begin{tabular}{ll}
    Paramètre & Diode réelle \\\hline
    Tension pointe de la charge & \(V_{Lp}=V_{2p}-V_{\mathrm{seuil}}\)\\
    Tension moyenne de la charge & \(V_{L\mathrm{moy}}=V_{2p}-V_{\mathrm{seuil}}/\pi \)\\
    Courant moyen dans la charge & \(I_{L\mathrm{moy}}=V_{2p}-V_{\mathrm{seuil}}/\pi R_{L} \)\\
    Courant moyen dans la diode &  \(I_{D\mathrm{moy}}=V_{2p}-V_{\mathrm{seuil}}/\pi R_{L} \)\\
    Tension inverse de crête & \(TIC=V_{2p}\)
\end{tabular}

\subsubsection{Double alternance}
\renewcommand{\arraystretch}{2}
\begin{tabular}{ll}
    Paramètre & Diode réelle \\\hline
    Tension pointe de la charge & \(V_{Lp}=\frac{V_{2p}}{2}-V_{\mathrm{seuil}}\)\\
    Tension Moyenne de la charge & \(V_{L\mathrm{moy}}=\frac{V_{2p}-2V_{\mathrm{seuil}}}{\pi}\)\\
    Courant moyen dans la charge & \(I_{L\mathrm{moy}}=\frac{V_{2p}-2V_{\mathrm{seuil}}}{\pi R_L}\)\\
    Courant moyen dans les diodes &  \(I_{D\mathrm{moy}}=\frac{V_{2p}-2V_{\mathrm{seuil}}}{2\pi R_L}\)\\
    Tension inverse de crête & \(TIC_{D1} = TIC_{D2}=V_{2p}\)
\end{tabular}

\subsubsection{en pont}
\begin{tabular}{ll}
    Paramètre & Diode réelle \\\hline
    Tension pointe de la charge & \(V_{Lp}=V_{2p}-2V_{\mathrm{seuil}}\)\\
    Tension moyenne de la charge & \(V_{L\mathrm{moy}}=\frac{2(V_{2p}-2V_{\mathrm{seuil}})}{\pi}\)\\
    Courant moyen dans la charge & \(I_{L\mathrm{moy}}=\frac{2(V_{2p}-2V_{\mathrm{seuil}})}{\pi R_L}\)\\
    Courant moyen dans les diodes &  \(I_{D\mathrm{moy}}=\frac{V_{2p}-2V_{\mathrm{seuil}}}{\pi R_L}\)\\
    Tension inverse de crête & \(TIC_{Di}=V_{2p}\)
\end{tabular}

\subsection{avec filtrage}
\renewcommand{\arraystretch}{1}
\subsubsection{Simple alternance $f=f_{\mathrm{in}}$}
\begin{center}
    \includestandalone[scale=1.25]{fig/redresseur_avec_filtrage}
\end{center}
\begin{tabular}{ll}
    Paramètre & Diode réelle \\\hline
    Tension pointe de la charge & \(V_{Lp}=V_{2p}-V_{\mathrm{seuil}}\)\\
    Tension moyenne de la charge & \(V_{L\mathrm{moy}}=V_{Lp}-(V_{R}/2)\)\\
    Ondulation de la tension & \(V_{R}=V_{Lp}/(R_L f C)\)\\
    Courant moyen dans la charge & \(I_{L\mathrm{moy}}=V_{L\mathrm{moy}}/R_{L} \)\\
    Courant moyen dans la diode &  \(I_{D\mathrm{moy}}=I_{L\mathrm{moy}}\)\\
    Courant maximum dans la diode & \(I_p=I_{D\mathrm{moy}}*2(V_{L\mathrm{moy}} / V_R)\)\\
    Tension inverse de crête & \(TIC=2V_{2p}\)
\end{tabular}

\subsubsection{Double alternance $f=2f_{\mathrm{in}}$}
\begin{tabular}{ll}
    Paramètre & Diode réelle \\\hline
    Tension pointe de la charge & \(V_{Lp}=\frac{V_{2p}}{2}-V_{\mathrm{seuil}}\)\\
    Tension moyenne de la charge & \(V_{L\mathrm{moy}}=V_{Lp}-(V_R /2)\)\\
    Ondulation de la tension & \(V_{R}=V_{Lp}/(R_L f C)\)\\
    Courant moyen dans la charge & \(I_{L\mathrm{moy}}=V_{L\mathrm{moy}}/R_{L} \)\\
    Courant moyen dans les diodes &  \(I_{D\mathrm{moy}}=I_{L\mathrm{moy}}/2\)\\
    Courant maximum dans la diode & \(I_p=I_{D\mathrm{moy}}*2(V_{L\mathrm{moy}} / V_R)\)\\
    Tension inverse de crête & \(TIC_{D1} = TIC_{D2}=V_{2p}\)
\end{tabular}

\subsubsection{en pont $f=2f_{\mathrm{in}}$}
\begin{center}
    \includestandalone[scale=1]{fig/pont_diode}
\end{center}
\begin{tabular}{ll}
    Paramètre & Diode réelle \\\hline
    Tension pointe de la charge & \(V_{Lp}=V_{2p}-2V_{\mathrm{seuil}}\)\\
    Tension moyenne de la charge & \(V_{L\mathrm{moy}}=V_{Lp}-(V_R /2)\)\\
    Ondulation de la tension & \(V_{R}=V_{Lp}/(R_L f C)\)\\
    Courant moyen dans la charge & \(I_{L\mathrm{moy}}= V_{L\mathrm{moy}} / R_L\)\\
    Courant moyen dans les diodes &  \(I_{D\mathrm{moy}}=I_{L\mathrm{moy}}/2\)\\
    Courant maximum dans la diode & \(I_p=I_{D\mathrm{moy}}*2(V_{L\mathrm{moy}} / V_R)\)\\
    Tension inverse de crête & \(TIC_{Di}=V_{2p}\)
\end{tabular}