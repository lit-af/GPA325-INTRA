\section{Circuits RLC}
\subsection{Type de régime}
\begin{tabular}{ll}
\(\omega_0^2 < \alpha^2 \rightarrow \zeta > 0\) & Sur-amorti\\
\(\omega_0^2 > \alpha^2 \rightarrow \zeta < 0 \) & Sous-amorti\\
\(\omega_0^2 = \alpha^2 \rightarrow \zeta = 0\) & Amortissement critique\\
\end{tabular}

\subsection{En série}
\begin{tabular}{llcl}
Racine  & \(s_{1,2}\) & \(=\) & \( -\alpha \pm \sqrt{\alpha^2-\omega_0^2} \) \\
Coeff. amortissement (rad/s) & \( \alpha\) & \(=\) & \(R/2L \) \\
Fréq. naturelle (rad/s) & \(\omega_0\) & \(=\) & \(1/\sqrt{LC} \) \\
Taux d'ammortissement & \( \zeta\) & \(=\) & \(\alpha/\omega_0\)
\end{tabular}

\subsubsection{Sur-amorti $\omega^2 < \alpha^2$}
\begin{align*}
i(t) &= I_f + A_1 e^{s_1 t}+A_2 e^{s_2t} \\
v(0) &= A_1+A_2  \\
i_C(0)/C &= s_1 A_1 + s_2 A_2
\end{align*}

\subsubsection{Sous-amorti $\omega^2 > \alpha^2$}
\begin{align*}
    i(t) &= I_f + B_1 e^{-\alpha t}\cos\qty(\omega_d t)+B_2 e^{-\alpha t}\sin\qty(\omega_d t)\\
    \omega_d &= \sqrt{\omega_0^2-\alpha^2}\\
    i(0) &= B_1 + I_f\\
    v_L(0)/L &= -\alpha B_1 + \omega_d B_2
\end{align*}

\subsubsection{Amortissement critique $\omega^2 = \alpha^2$}
\begin{align*}
i(t) &= I_f + D_1 t e^{-\alpha t}+D_2 e^{-\alpha t}\\
i(0) &= I_0 = D_2\\
v_L(0)/L &= D_1 -\alpha D_2
\end{align*}

\subsection{En parallèle}
\begin{tabular}{llcl}
Racine  & \(s_{1,2}\) & \(=\) & \(-\alpha \pm \sqrt{\alpha^2-\omega_0^2} \) \\
Coeff. amortissement (rad/s)& \( \alpha\) & \(=\) & \(1/2RC \) \\
Fréq. naturelle (rad/s) & \(\omega_0\) & \(=\) & \(1/\sqrt{LC}\) \\
Taux d'ammortissement & \( \zeta\) & \(=\) & \(\alpha/\omega_0\)
\end{tabular}

\subsection{Équations des courants}
\begin{align*}
    i_R(t) &= v(t)/R \\
    i_C(t) &= C\frac{dv(t)}{dt}\\
    i_L(t) &= -i_R(t) - i_C(t)
\end{align*}

\subsubsection{Sur-amorti $\omega^2 < \alpha^2$}
\begin{align*}
v(t) &= V_f + A_1 e^{s_1 t}+A_2 e^{s_2t} \\    
v(0) &= A_1+A_2  \\
i_C(0)/C &= s_1 A_1 + s_2 A_2
\end{align*}

\subsubsection{Sous-amorti $\omega^2 > \alpha^2$}
\begin{align*}
    v(t) &= V_f + B_1 e^{-\alpha t}\cos\qty(\omega_d t)+B_2 e^{-\alpha t}\sin\qty(\omega_d t)\\
    \omega_d &= \sqrt{\omega_0^2 - \alpha^2}\\
    v(0) &= V_0 = B_1 \\
    i_C(0)/C &= -\alpha B_1+\omega_d B_2
\end{align*}

\subsubsection{Amortissement critique $\omega^2 = \alpha^2$}
\begin{align*}
    v(t) &= V_f + D_1 t e^{-\alpha t} + D_2 e^{-\alpha t}\\
    v(0) &= V_0 = D_2\\
    i_C(0)/C &= D_1 - \alpha D_2
\end{align*}